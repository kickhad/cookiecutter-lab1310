\documentclass[12pt]{article}
\usepackage{amsfonts, amsmath, amsthm, amssymb}
\usepackage{graphicx}
\usepackage{xcolor}
\usepackage[letterpaper, margin=1in]{geometry}
\usepackage{heuristica}
\usepackage[heuristica,vvarbb,bigdelims]{newtxmath}
\usepackage[T1]{fontenc}
\renewcommand*\oldstylenums[1]{\textosf{#1}}
\usepackage{listings}
\usepackage{enumitem}
\usepackage{color} %red, green, blue, yellow, cyan, magenta, black, white
\definecolor{mygreen}{RGB}{28,172,0} % color values Red, Green, Blue
\definecolor{mylilas}{RGB}{170,55,241}



	


\begin{document}
\vspace*{6cm}
\begin{center}	{\Large\bfseries {{cookiecutter.project_name}} - {{cookiecutter.project_decription}}}

\vspace*{0.7cm}
\begin{description}[font=$\bullet$~\bfseries]
\item [{{cookiecutter.author}}]
\item [{{cookiecutter.lab_section}}]
\item [{{cookiecutter.student_number}}]

\end{description}

\end{center}


\pagebreak
\lstset{language=Matlab,%
    %basicstyle=\color{red},
    breaklines=true,%
    morekeywords={matlab2tikz},
    keywordstyle=\color{blue},%
    morekeywords=[2]{1}, keywordstyle=[2]{\color{black}},
    identifierstyle=\color{black},%
    stringstyle=\color{mylilas},
    commentstyle=\color{mygreen},%
    showstringspaces=false,%without this there will be a symbol in the places where there is a space
    numbers=left,%
    numberstyle={\tiny \color{black}},% size of the numbers
    numbersep=9pt, % this defines how far the numbers are from the text
    emph=[1]{for,end,break},emphstyle=[1]\color{red}, %some words to emphasise
    %emph=[2]{word1,word2}, emphstyle=[2]{style},    
}

                                                       
\section*{Problem 1}                                                     
\subsection*{Deliverable }
\subsubsection*{}
\lstinputlisting{../func.m}


% \begin{figure}
%   \includegraphics[width=\linewidth]{func3.jpg}
%   \caption{Analytical 3}
%   \label{fig:func3}
% \end{figure}

\end{document}